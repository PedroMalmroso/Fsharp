\documentclass[a4paper]{article}

%%%%%%%%%%%%%%%%%%%%%%%%%%%%%%%%%%
% Package for making LaTeX properly handle utf8 characters set and danish language rules
\usepackage[utf8]{inputenc}
\usepackage[danish]{babel}

%%%%%%%%%%%%%%%%%%%%%%%%%%%%%%%%%%
% Package for changing to a nicer font 
\usepackage [T1]{fontenc}

%%%%%%%%%%%%%%%%%%%%%%%%%%%%%%%%%%
% Package for conctroling the text area
\usepackage[margin=2.5cm]{geometry}

%%%%%%%%%%%%%%%%%%%%%%%%%%%%%%%%%%
% Package for inserting clickable hyperlinks in pdf versions as produced by pdflatex
\usepackage{hyperref}

%%%%%%%%%%%%%%%%%%%%%%%%%%%%%%%%%%
% Package for including figures. TeX and thus LaTeX was developped before the existence of directory file-structures, but the graphicspath let's you add directories, that the \includegraphics will search.
\usepackage{graphicx}
\graphicspath{{figures/}{anotherFigureDirectory/}}

%%%%%%%%%%%%%%%%%%%%%%%%%%%%%%%%%%
% Package for typesetting programs. Listings does not support fsharp, but a little modification goes a long way
\usepackage{listings}
\usepackage{color}

\definecolor{bluekeywords}{rgb}{0.13,0.13,1}
\definecolor{greencomments}{rgb}{0,0.5,0}
\definecolor{turqusnumbers}{rgb}{0.17,0.57,0.69}
\definecolor{redstrings}{rgb}{0.5,0,0}

\lstdefinelanguage{FSharp}
                {morekeywords={let, new, match, with, rec, open, module, namespace, type, of, member, and, for, in, do, begin, end, fun, function, try, mutable, if, then, else},
    keywordstyle=\color{bluekeywords},
    sensitive=false,
    morecomment=[l][\color{greencomments}]{///},
    morecomment=[l][\color{greencomments}]{//},
    morecomment=[s][\color{greencomments}]{{(*}{*)}},
    morestring=[b]",
    stringstyle=\color{redstrings}
    }
%%%%%%%%%%%%%%%%%%%%%%%%%%%%%%%%%%
% Package for extended math settings, e.g. \eqref
\usepackage{amsmath}

%%%%%%%%%%%%%%%%%%%%%%%%%%%%%%%%%%
% These will be the title and author, as included when \maketitle is called.
\title{Programmering \& Problemløsning | opgave $1.i$}
\author{Peter Trip Malmroes (WBL189) }
%\date{september 2023}

\begin{document}
\maketitle % Insert title etc.

\section*{Opgave $1.i.0$}

Denne opgave definerer en rekursiv funktion, som bruger et match-with statement. Match-with statementet matcher brugerinput med forskellige handlinger. Der sørges for at programmet kalder sig selv, sådan at det i terminalen bliver ved med at tage brugerinput indtil brugeren skriver quit, hvorved \verb|exit| \verb|0| kaldes. 

\section*{Opgave $1.i.1$}

I denne opgave bruges funktionen \verb|interact| til at interagere en react-funktion der mapper piletasterne til ændringer i en state x som er x-koordinaten i boksen. React-funktionen interageres med draw-funktionen, som tager x-koordinaten som input. Draw-funktionen laver en hvid rektangel der fylder hele vinduets størrelse, og placerer en sort rektangel ovenpå via \verb|onto|. Via variablen x fra react-funktionen og \verb|translate| rykkes den sorte rektangel horisontalt. De fire linjer dannes via \verb|piecewiseAffine| som også tager state x som input i slutpunktet for de respektive linjer. Alle linjer og den sorte boks skaleres relativt til $w,h$ sådan at de dynamisk tilpasses $w,h$ hvis disse ændres. Nedenfor i figur \ref{fig: Output} ses outputtet af interaktionen. Jeg løste ikke den åbne opgave (d), men jeg antager at man skal lave state om til en tuple, som også tager et output fra mellemrumstasten i react-funktionen med ind i draw-funktionen som så eksekverer \verb|RenderToFile| hvis mellemrumstasten trykkes.

\begin{figure}[h!]
  \includegraphics[width=\linewidth]{Billeder/1i.png}
  \caption{Output}
  \label{fig: Output}
\end{figure}

\end{document}

